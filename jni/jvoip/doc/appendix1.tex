\chapter*{Appendix A\\\vspace{0.9cm}Why `Destroy' must be called explicitly}
\addcontentsline{toc}{chapter}{Appendix A: Why 'Destroy' must be called explicitly}

In the versions prior to 1.2.0, the destructor of {\tt JVOIPSession} simply
called the {\tt Destroy} method. David Allouche pointed out that this can 
lead to all sorts of problems. For
this reason, it is now illegal to call the destructor of a session which
is still active. The session must first be terminated by the user, using
the {\tt Destroy} function.
In this appendix I will try to explain why problems can arise with the
previously used approach. 

The core of the problem is the use of virtual functions. For example,
the {\tt Destroy} method calls the virtual function {\tt UserDefinedDestroy}.
Consider the following example:

	\begin{lstlisting}[frame=tb]{}
class MySession : public JVOIPSession
{
public:
	MySession() { mydata = NULL; }
	~MySession() { }
	bool UserDefinedCreate()
	{
		mydata = new MyData();
		return true;
	}
	void UserDefinedDestroy()
	{
		delete mydata;
		mydata = NULL;
	}
}
\end{lstlisting}

In the previous example, we want some memory to be allocated when the session is 
created and we would like that memory to be returned to the system when the 
session is destroyed. Now, in the previously used approach the {\tt JVOIPSession}
destructor would simply call {\tt Destroy}. Let's take a look at what would happen.

Suppose that at some point you have an active session and call the {\tt MySession}
destructor, for example by doing something like: {\tt delete mysession; }
First, the {\tt MySession} destructor would be executed (which does nothing in this
case). Then, the {\tt JVOIPSession} destructor is called, which in turn calls the
{\tt Destroy} member function. At some point, {\tt Destroy} executes the virtual
function {\tt UserDefinedDestroy}.

However, at that moment the {\tt MySession} part of the object does not exist anymore.
Normally the virtual function pointer of {\tt UserDefinedDestroy} should have been
reset to the address of {\tt JVOIPSession}'s implementation of this function. So,
instead of calling the {\tt User\-Defined\-Destroy} function of {\tt MySession},
the {\tt JVOIPSession} method is called instead. This obviously will cause a
memory leak since the memory allocated in {\tt MySession}'s {\tt User\-Defined\-Create}
function is not released when the object is destroyed.

You may argue that having one memory leak is not that bad, but this example simply
illustrates that unexpected things can (and will) happen with that approach.
Also, keep in mind that the library uses many virtual functions (including
pure virtual functions which, by definition, have no default implementation)
in several threads. This only makes it more likely that something can go wrong
that way. For this reason, calling the destructor of an active session will
now cause an exception to be thrown.

