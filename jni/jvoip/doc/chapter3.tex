\chapter{Using the library}
In this chapter I'll explain how you can use the library. First I'll
give some more information about how the {\tt JVOIPSession} class can
be used. Then, I'll give some more information about the implemented
components.

	\section{A simple VoIP session}
	
	The library is written in such a way that you can create a basic
	VoIP session with only a few lines of code. The following example
	would create a VoIP session, wait for the user to give some input
	and destroy the VoIP session.
	\begin{lstlisting}[frame=tb]{}
#include "jvoipsession.h"
#include <stdio.h>

int main(void)
{
	JVOIPSession session;
	JVOIPSessionParams params;

	session.Create(params);
	fgetc(stdin);
	session.Destroy();

	return 0;
}
	\end{lstlisting}
	
	However, if you try this you won't see (or hear) much happening. The
	reason for this is that we haven't specified a destination yet.
	
	In the example above, we simply specified an instance of {\tt JVOIP\-Session\-Params}
	as the session parameters. We didn't specify anything ourselves, so
	we're using the default parameters. These are:
	\begin{itemize}
		\item soundcard input \& soundcard output
		\item no localisation scheme
		\item no compression
		\item the default mixer
		\item the RTP transmission module
		\item accept all incoming packets
		\item eight bits per sample
		\item 20 ms sample interval
		\item 8000 Hz input sampling rate
		\item 8000 Hz output sampling rate
	\end{itemize}
	The component parameters are all set to {\tt NULL}, which will cause the
	components to use their default parameters.
	
	Now, suppose we would like to make an application that sends data to
	an address specified by the user, at port 5000. If two such applications
	are running on different hosts and they would be sending data to each
	other, you have your first working VoIP program\footnote{The default
	portbase of the RTP transmission module is 5000}.
	
	The example would become something like this:
	\begin{lstlisting}[frame=tb]{}
#include "jvoipsession.h"
#include <stdio.h>
#include <arpa/inet.h>
#include <netinet/in.h>

int main(void)
{
	JVOIPSession session;
	JVOIPSessionParams params;
	char str_addr[255];
	unsigned long ip;

	printf("Enter the destination address:\n");
	scanf("%s",str_addr);
	fgetc(stdin); // read the current end-of-line character
	ip = inet_addr(str_addr);
	ip = ntohl(ip);

	session.Create(params);
	session.AddDestination(ip,5000);

	fgetc(stdin);
	session.Destroy();

	return 0;
}

	\end{lstlisting}
	
	\section{Error handling}
	
		\subsection{Failed function calls}
		
		Most functions of the library return an {\tt int}. If these functions
		are successfull, they have a return value of zero or more. Otherwise,
		they return a negative value. 
		
		When an error did occur, you can always call the {\tt JVOIP\-Session} 
		member function {\tt IsActive} to find out if the error has caused the destruction 
		of the session. The library does its best to avoid this, but in some
		cases it's not possible to recover from an error.
		
		The library contains a function called {\tt JVOIP\-Get\-Error\-String}
		which takes an error code as its argument. The function then returns
		a string which describes the error. Its syntax is this:
		\begin{lstlisting}[frame=tb]{}
std::string JVOIPGetErrorString(int errcode);
		\end{lstlisting}
		
		In the previous example, we did not perform checks on the return value
		of the called functions. Suppose we'd define a function {\tt CheckError}
		like this:
		\begin{lstlisting}[frame=tb]{}
void CheckError(int val)
{
	if (val >= 0)
		return;
		
	std::string errstr;
	
	errstr = JVOIPGetErrorString(val);
	std::cerr << errstr << std::endl;
	exit(-1);
}
		\end{lstlisting}
		
		Then, we could easily check for errors by changing a part of the code
		of the example program:
		\begin{lstlisting}[frame=tb]{}
	CheckError(session.Create(params));
	CheckError(session.AddDestination(ip,5000));
	
	fgetc(stdin);
	CheckError(session.Destroy());
		\end{lstlisting}
		
		\subsection{Errors in the VoIP thread}
		
		Now, what if an error occurs in the VoIP thread? Since this is a background
		process, you can't just check the return value of some function of the 
		library's API.
		
		Fortunately, there is a way to check this. As was already explained in 
		\ref{subsection-overridables}, the {\tt JVOIPSession} member function
		{\tt Thread\-Finished\-Handler} is called whenever the VoIP thread ends.
		The parameters of this function contain information about why the thread
		stopped. By deriving your own class from {\tt JVOIP\-Session} and overriding
		the function {\tt Thread\-Finished\-Handler}, you can check when and why the
		thread ended.
		
		The syntax of the {\tt ThreadFinishedHandler} function is as follows:
		\begin{lstlisting}[frame=tb]{}
void ThreadFinishedHandler(int threaderr,
                           int voicecallerr = 0,
                           int componenterr = 0)
		\end{lstlisting}
		
		The first argument, {\tt threaderr}, specifies why the thread ended.
		There are two possibilities: either an error occurred in the {\tt Voice\-Call}
		member function {\tt Step}, or the {\tt JVOIPSession} member function
		{\tt IntervalAction} (which you can override in a derived class) returned
		{\tt false}. In the first case, {\tt threaderr} contains the value 
		{\tt ERR\_\-JVOIP\-LIB\_\-VOIP\-THREAD\_\-VOICE\-CALL\-ERROR}. In the second
		case, the value will be {\tt ERR\_\-JVOIP\-LIB\_\-VOIP\-THREAD\_\-INTER\-VAL\-ACTION\-ERROR}.
		
		If the thread ended because {\tt IntervalAction} returned {\tt false},
		the other arguments of {\tt Thread\-Finished\-Handler} will be zero.
		Otherwise, they contain more information about what caused the error.
		
		The {\tt voicecallerr} parameter contains information about where in the
		{\tt Step} function the error occurred. The possible values are defined
		in the file {\tt src/\-component\-framework/\-voice\-call.h}. Note that
		you cannot use the {\tt JVOIPGetErrorString} function to get a description
		of this error. This is because the VoIP framework is a separate part of
		the library. To get a description of this error code, you can call the
		{\tt GetVoiceCallErrorString} function (in the VoIPFramework namespace).
		
		If the error was caused by a specific component, {\tt com\-ponent\-err}
		specifies the error of the specific component. If the component is {\em not}
		a user-defined component, you can get a description of the error by calling
		the {\tt JVOIPGetErrorString} function with {\tt componenterr} as its
		argument.
		
	\section{Setting component parameters}
	Suppose we would like to create an easy test application: one that sends
	data to itself. You could try to use the previous program and fill in the
	local host address {\tt 127.0.0.1}, but it won't work. This is so because
	by default the RTP transmission module is used and this module ignores
	its own packets by default\footnote{This comes in handy when using multicasting}.
	
	Fortunately, you can instruct the RTP module to accept its own packets.
	This can be done by selecting this feature in the module's parameters.
	More concrete, we'll have to add a piece of code like this:
	\begin{lstlisting}[frame=tb]{}
JVOIPRTPTransmissionParams rtpparams;

rtpparams.SetAcceptOwnPackets(true);
params.SetTransmissionParams(&rtpparams);
	\end{lstlisting}
	Of course, this has to be done before the session parameters are used in
	the {\tt Create} function.
	
	Our application code would then become something like this:
	\begin{lstlisting}[frame=tb]{}
#include "jvoipsession.h"
#include "jvoiprtptransmission.h"
#include <stdio.h>
#include <arpa/inet.h>
#include <netinet/in.h>

int main(void)
{
	JVOIPSession session;
	JVOIPSessionParams params;
	JVOIPRTPTransmissionParams rtpparams;
	unsigned long ip;

	ip = inet_addr("127.0.0.1");
	ip = ntohl(ip);

	rtpparams.SetAcceptOwnPackets(true);
	params.SetTransmissionParams(&rtpparams);

	session.Create(params);
	session.AddDestination(ip,5000);
	fgetc(stdin);
	session.Destroy();

	return 0;
}
	\end{lstlisting}
	
	For other components, parameters can be passed in a similar way.
	
	\section{Changing features of an active session}
	
	Once the session is active, you can still change all of its features. The
	class {\tt JVOIPSession} has several functions to do this. For example, if
	you would like to change your sample interval to 50 ms, you simply have
	to add a line:
	\begin{lstlisting}[frame=tb]{}
session.SetSampleInterval(50);
	\end{lstlisting}
	
	Similarly, you can change a component type. Suppose that you would like to
	change from a soundcard output type to no output at all, you would do
	something like this:
	\begin{lstlisting}[frame=tb]{}
session.SetVoiceOutputType(JVOIPSessionParams::NoOutput,NULL);
	\end{lstlisting}
	
	The same procedure has to be followed if you would like to change the
	component parameters of the current component. There is no general way
	to instruct a component to change a specific parameter, but you can
	reinstall the current component type with different parameters.
	
	For example, suppose we would like to change the portbase of our RTP
	transmission module to 6000. This is the way to do it:
	\begin{lstlisting}[frame=tb]{}
rtpparams.SetPortBase(6000);
session.SetTransmissionType(JVOIPSessionParams::RTP,&rtpparams);
	\end{lstlisting}
	If you do something like this, keep in mind that you have created a new
	instance of a module. In the example above, this would cause the loss
	of all previously set destinations, so you'll have to add those again.
	
	\section{Component parameters}
	
	Lets take a closer look at the available parameters for the currently
	implemented components.
	
		\subsection{{\tt JVOIPSoundcardInput} component}
		
		First, lets take a look at the soundcard input module's parameters.
		
			\subsubsection{Unix version}
			
			In the unix version, the soundcard module uses a specific soundcard
			device. This device can be specified by passing an instance of
			{\tt JVOIPSoundcardParams} to the component. By default, the device
			{\tt /dev/dsp} is used:
			
			\begin{lstlisting}[frame=tb]{}
class JVOIPSoundcardParams : public JVOIPComponentParams
{
public:
	JVOIPSoundcardParams(const std::string &dev = "/dev/dsp", 
	                     int multiplyfactor = 1);
	void SetSoundDeviceName(const std::string &n);
	void SetMultiplyFactor(int multiplyfactor);
};
			\end{lstlisting}
			The multiply factor specifies the value with which samples read from
			the soundcard should be multiplied.
			
			\subsubsection{MS-Windows version}
			
			When using MS-Windows, we can make use of its Win32 soundcard interface
			and we do not need to specify a device like with unix. In this case,
			we don't really need to pass component parameters.
			
			However, when experimenting with VoIP applications, I noticed that
			sometimes -- probably because of a bad soundcard driver -- when you
			do a reset of the soundcard input, suddenly the input settings are
			messed up. If this should be the case, you can try a bugfix I wrote.
			
			To be safe, the bugfix is turned on by default, but if you think
			you don't need it, you can turn it off by disabling the corresponding
			feature in the soundcard input parameters:
			
			\begin{lstlisting}[frame=tb]{}
class JVOIPSoundcardInputParams : public JVOIPComponentParams
{
public:
	JVOIPSoundcardInputParams(bool resetbugfix = true);
	void SetNastyResetBugFix(bool resetbugfix);
};
			\end{lstlisting}
			
		\subsection{{\tt JVOIPSoundcardOutput} component}
		
		For the soundcard output parameters, there is only a unix version.
		Like with the soundcard input on a unix-like OS, you need to
		specify a device. This can be done the same way as with the soundcard
		input parameters, using the class {\tt JVOIPSoundcardParameters}.

		\subsection{{\tt JVOIPFileInput} component}

		The file input component takes an instance of {\tt JVOIPFileInputParams}
		as its parameters. The constructor of this class is as follows:

		\begin{lstlisting}[frame=tb]{}
class JVOIPFileInputParams : public JVOIPComponentParams
{
public:
		JVOIPFileInputParams(const std::string &filename)
};		
		\end{lstlisting}
		This way, you can specify which file should be used for input. The file
		input component uses {\tt libsndfile}\footnote{\url{http://www.mega-nerd.com/libsndfile/}}
		to read from sound files.
		
		\subsection{{\tt JVOIPFileOutput} component}

		The library contains a component to write to a sound file, which also
		makes use of the {\tt libsndfile} library. The parameters it accepts
		are described in the {\tt JVOIPFileOutputParams} class:
		\begin{lstlisting}[frame=tb]{}
class JVOIPFileOutputParams : public JVOIPComponentParams
{
public:
	enum EncodingType { SixteenBit, EightBit, MuLaw, Other };
	
	JVOIPFileOutputParams(const std::string &filename,
	                      int samplerate = 8000,
	                      EncodingType t = MuLaw,
	                      bool stereo = false);
	void SetFileName(const std::string &filename);
	void SetSampleRate(int samplerate);
	void SetStereo(bool stereo);
	void SetEncodingType(EncodingType t);
};
		\end{lstlisting}
		If the file already exists, it is opened with the parameters of the file itself 
		and data is appended to that file. When a new file needs to be created, the parameters
		specified in the class above are used.

		\subsection{{\tt JVOIPDPCMCompression} component}
		
		The DPCM compression module can be instructed to use a specific amount
		of encoding bits (for the differences). By default, the number of
		encoding bits is five:
		
		\begin{lstlisting}[frame=tb]{}
class JVOIPDPCMCompressionParams : public JVOIPComponentParams
{
public:
	JVOIPDPCMCompressionParams(int encodebits = 5);
	void SetNumEncodeBits(int n);
};
		\end{lstlisting}
		
		\subsection{{\tt JVOIPRTPTransmission} component}
		
		The RTP transmission module has several parameters, as you can see
		in the definition of the {\tt JVOIPRTPTransmissionParams} class below.
		You can set these parameters:
		\begin{itemize}
			\item {\bf The portbase}\\
			      This defines at which ports the module will poll for
			      incoming RTP and RTCP packets. By default, the portbase
			      is 5000.
			\item {\bf Auto adjust buffering}\\
			      If this feature is enabled, the module will determine
			      automatically how much buffering is required to eliminate
			      jitter. By default, this feature is enabled.
			\item {\bf Default buffering}\\
			      If `auto adjust buffering' is not enabled, this specifies
			      how much buffering is inserted into each RTP stream. By
			      default, this is 50 ms.
			\item {\bf Minimum buffering}\\
			      This specifies the minimum amount of buffering to be inserted
			      in each RTP stream. By default this is 20 ms. This feature is
			      especially useful when `auto adjust buffering' is enabled. In 
			      that case it is still possible that the amount of buffering is
			      too low for some streams. This way, you can set a minimum.
			\item {\bf Accept own packets}\\
			      By default, the RTP module ignores its own packets. If you
			      want to accept them, you'll have to enable this feature.
			\item {\bf Set local IP address}\\
			      The RTP module tries to determine your IP address automatically.
			      If, for some reason, this IP address should be wrong, you can
			      specify it yourself with this parameter. By default it is set to
			      zero, which means that the RTP module will try to detect your IP
			      address.
		\end{itemize}
		
		
		\begin{lstlisting}[frame=tb]{}
class JVOIPRTPTransmissionParams : public JVOIPComponentParams
{
public:
	JVOIPRTPTransmissionParams(int pbase = 5000,
	                           bool autobuffer = true,
	                           double defbuffering = 0.050,
	                           double minbuffering = 0.020,
	                           bool acceptownpack = false,
	                           unsigned long locIP = 0);
	void SetPortBase(int pb);
	void SetAutoAdjustBuffer(bool autobuffer);
	void SetDefaultBuffering(double defbuffer);
	void SetMinimumBuffering(double minbuffer);
	void SetAcceptOwnPackets(bool acceptownpack);
	void SetLocalIP(unsigned long locIP);
};
		\end{lstlisting}
	
	\section{Overriding {\tt JVOIPSession} member functions}
	
	If you derive your own class from {\tt JVOIP\-Session}, you can use some
	extra features of the library by overriding some of the library's
	member functions. An example presented earlier was the use of the 
	{\tt ThreadFinishedHandler}.
	
		\subsection{User defined initialization and destruction}
		
		When you derive your class from {\tt JVOIP\-Session}, you may want to
		perform some extra initialization when a VoIP session is created.
		This can be done by overriding the {\tt UserDefinedCreate} function.
		Its syntax is like this:
		\begin{lstlisting}[frame=tb]{}
bool UserDefinedCreate()
		\end{lstlisting}
		This function should return {\tt true} if your initialization was
		successfull or {\tt false} in case an error occurred.
		
		Similarly, there is also a function which is called when the VoIP
		session is destroyed. If you want something to be done at that point,
		you can override {\tt UserDefinedDestroy}, which is defined like
		this:
		\begin{lstlisting}[frame=tb]{}
void UserDefinedDestroy()
		\end{lstlisting}
		
		\subsection{User-defined components}
		
		If you want to register a user-defined component, you have to override
		one of the following functions, depending on the component type:
		\begin{lstlisting}[frame=tb]{}
bool RegisterUserDefinedInput(JVOIPVoiceInput **i)
bool RegisterUserDefinedOutput(JVOIPVoiceOutput **o)
bool RegisterUserDefinedLocalisation(JVOIPLocalisation **l)
bool RegisterUserDefinedCompressionModule(JVOIPCompressionModule **c)
bool RegisterUserDefinedTransmission(JVOIPTransmission **t)
bool RegisterUserDefinedMixer(JVOIPMixer **m)
		\end{lstlisting}
		
		In these functions, you fill in a pointer to your component in the argument. 
		If the registration was successfull, you have to return {\tt true},
		otherwise, {\tt false}.
		
		For example, suppose we have made our own voice output module, called 
		{\tt My\-Output\-Module}, and that we're deriving a class {\tt My\-Session}
		from {\tt JVOIP\-Session}. In this case, we could do something like this:
		\begin{lstlisting}[frame=tb]{}
bool MySession::RegisterUserDefinedOutput(JVOIPVoiceOutput **o)
{
	MyOutputModule *m;

	m = new MyOutputModule();
	if (m == NULL)
		return false;
	*o = m;
	return true;
}
		\end{lstlisting}
		
		Similar to the {\tt Register\-User\-DefinedX} functions, there are also functions
		to unregister your component. These functions are called when the session is
		destroyed and can be used to perform some cleanup operations. The functions
		look like this:
		\begin{lstlisting}[frame=tb]{}
void UnregisterUserDefinedInput(JVOIPVoiceInput *i)
void UnregisterUserDefinedOutput(JVOIPVoiceOutput *o)
void UnregisterUserDefinedLocalisation(JVOIPLocalisation *l)
void UnregisterUserDefinedCompressionModule(JVOIPCompressionModule *c)
void UnregisterUserDefinedTransmission(JVOIPTransmission *t)
void UnregisterUserDefinedMixer(JVOIPMixer *m)
		\end{lstlisting}
		
		In the example above, we dynamically allocated our own output module.
		To prevent memory leaks, we should destroy this module in the
		{\tt Unregister\-User\-Defined\-Output} routine:
		\begin{lstlisting}[frame=tb]{}
void MySession::UnregisterUserDefinedOutput(JVOIPVoiceOutput *o)
{
	if (o != NULL)
		delete o;
}
		\end{lstlisting}
		
		\subsection{\tt IntervalAction}
		
		Each time a sample interval passes, the {\tt IntervalAction} member function
		is called. The syntax of this function is as follows:
		\begin{lstlisting}[frame=tb]{}
bool IntervalAction(bool *waitforsignal)
		\end{lstlisting}
		
		It is possible that in your {\tt IntervalAction} implementation, you do
		something which clears the buffers of the signal waiter associated with
		the timing module. In that case, the VoIP thread would block if it waits
		for a signal (which is what the VoIP thread does after the call to
		{\tt Interval\-Action} is done). In this situation, you can avoid the 
		blocking of the thread by setting the {\tt wait\-for\-signal} argument 
		to {\tt false}.
		
		The {\tt JVOIPSession} class already has an implementation of {\tt Interval\-Action}
		in which it is checked if the connection is getting out of sync. If this
		happens it resets some modules, which improves the synchronization.
		
		This means that if you override the {\tt Interval\-Action} member function,
		this check is no longer performed. So it might be a good idea to make a call
		to {\tt JVOIP\-Session::\-Interval\-Action} from your own implementation of
		the function.
		
		\subsection{3D Localisation functions}
		
		The 3D localisation modules which are a part of the library, all call the
		following overridable members functions:
		\begin{lstlisting}[frame=tb]{}
bool RetrieveOwnPosition(double *xpos,double *ypos,double *zpos,
                         double *righteardir_x,double *righteardir_y,
                         double *righteardir_z,double *frontdir_x,
						 double *frontdir_y, double *frontdir_z,
						 double *updir_x,double *updir_y,
						 double *updir_z)
bool EncodeOwnPosition(unsigned char encodebuffer[], int *len)
bool DecodePositionalInfo(unsigned char buffer[], int len, 
                          double *xpos,double *ypos,double *zpos)
		\end{lstlisting}
		
		Only if these functions return {\tt true}, the information they provided
		is used. If they return {\tt false}, the session continues, but 
		no 3D effect will be added.
		
		In the arguments of {\tt Retrieve\-Own\-Position}, you should fill in both
		your own position in the 3D environment and the vector which contains
		the direction of your right ear.
		
		In {\tt Encode\-Own\-Position} you have to encode your own position into
		{\tt encode\-buffer}, which is a buffer of length {\tt JVOIP\_\-3DINFO\-BUFFER\-LEN}.
		The actual amount of bytes you used, has to be filled in in the agument
		{\tt len}.
		
		The {\tt Decode\-Positional\-Info} function has to be able to decode
		the positional info in {\tt buffer} and has to fill in the resulting 
		position in the 3D environment in the arguments {\tt xpos}, {\tt ypos} 
		and {\tt zpos}.
		
